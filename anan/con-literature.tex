\documentclass[11pt, oneside, reqno]{amsbook}

\usepackage[numbers,sort&compress]{natbib}
\usepackage{amsmath,amssymb,amsfonts,bbm}
\usepackage{algorithmic}
\usepackage{graphicx,float}
\usepackage{tabularray}
\usepackage{xcolor}
\usepackage[a4paper, margin=1.05in]{geometry}

\usepackage{titling}
\usepackage{conliteraturemacros}



\title{Literature for conference paper}
\author{Seangleng Khe, Parin Chaipunya}


\begin{document}

\begin{center}
  \huge{\bfseries\thetitle} \\[1em]
  \Large\theauthor
\end{center}
\vspace{10em}

\begingroup
\let\clearpage\relax
\tableofcontents
\endgroup

\chapter{Chance-constrained stochastic programming under variable reliability levels with an application to humanitarian relief network design}


\begin{longtblr}[caption={Summary of Model of \cite{elci2018}}]{@{}ll@{}}
  \hline\hline
  {\SetCell{l} \bfseries PART 1: Theoretical Parts} \\
  \hline
  {\SetCell{l} \bfseries Objective} \\
  $\min \leftarrow$ the risk tolerance \\
  \hline\hline
  {\bfseries Reformulation (in Value-at-risk): "equivalence relation"} \\
  $\mathbb{P} (T_k \textbf{x} \ge \xi_k ) \ge 1 - \epsilon_k \Leftrightarrow T_k \textbf{x} \ge \textbf{F}^{(-1)}_{\xi_k} (1 - \epsilon_k )$ \\
  \hline\hline
  {\bfseries Some theories used in the paper (Lemma 1)} \\ 
  Variable reduction \\
  Valid inequality
  \\ \hline \hline
  {\bfseries PART 2: Modeling with application to humanitarian relief network design} \\
  \hline 
  {\bfseries Objective} 
  $\min \leftarrow$ expected total delivery amount-weighted accessibilty score \\
  and expected cost of tolerance (s.t. 11 constraints)\\
  \hline \hline
\end{longtblr}



\chapter{A Chance-Constrained Programming (CCP) Approach to Solve the Energy Management Problem in Microgrids Considering Uncertainties of Renewable Energy Resources}

\begin{longtblr}[caption={Summary of Model of \cite{hojjat2024}}]{@{}ll@{}}
  \hline\hline
  {\SetCell{l} \bfseries Parameter and Variables} \\
  energy price \\
  revenue function \\
  cost function \\
  Total demanded energy \\
  Solar irradiance \\
  Risk to reward \\
  The energy level of the battery storage \\
  Probability density function\\
  Binary variables for charging of storage devices\\
  Binary variables for on or off status of generators \\
  \hline
  {\SetCell{l} \bfseries Objective} \\
  $\min \leftarrow$ the profit for the microgrid \\
  \hline\hline
  {\bfseries Reformulation} \\
  Sample Average Approximation (SAA)  \\
  \hline\hline
\end{longtblr}


\chapter{Analytical Reformulation of Chance-Constrained Optimal Power Flow with Uncertain Load Control}

\begin{longtblr}[caption={Summary of Model of \cite{li2023}}]{@{}ll@{}}
  \hline\hline
  {\SetCell{l} \bfseries Objective} \\
  $\min \leftarrow$ energy and reserve cost \\
  s.t. 6 deterministic (Secondary frequency control + Power flow constraints) \\
  and 19 prob constraints (re-dispatch constraints) \\
  \hline\hline
  {\SetCell{l} \bfseries Uncertainty: Gaussian} \\
  \hline \hline
  {\SetCell{l} \bfseries Reformulation} \\
  \hline
  {\SetCell{l} \bfseries Secondary frequency control constraints} \\
  $\bullet$ one-variable constraints $\rightarrow$ linear constraints \\
  $\bullet$ two-variable constraints $\rightarrow$ {\color{red} 2-norms!!} \\
  {\SetCell{l} \bfseries Re-dispatch} \\
  $\bullet$ re-define some variables \\
  $\bullet$ observe secondary frequency control: whether it is active or inactive. \\
  {\SetCell{l} \bfseries Power flow constraints} \\
  Most difficult as all uncertainties lie in these constraints\\
  \hline \hline
  {\SetCell{l} \bfseries Algorithm} \\
  Cutting plane (replace some complicated variables with slack variables)
\end{longtblr}

\chapter{On deterministic reformulations of distributionally robust joint chance constrained optimization problems}

\begin{longtblr}[caption={Summary of Model of \cite{xie2017deterministic}}]{@{}ll@{}}
  \hline\hline
  {\SetCell{l} \bfseries DRCCP} \\
  $\min c^T x ,$ \\
  s.t. $\inf_{\mathbb{P} \in \mathcal{P}} \mathbb{P} [ \xi: F(x, \xi) \ge 0] \ge 1 - \epsilon$ \\
  \hline\hline
  {\SetCell{l} \bfseries Method} \\
  $\bullet$ Define the feasible region $Z_D = \{ x \in \mathbb{R}^n: \inf_{\mathbb{P} \in \mathcal{P}} \mathbb{P} [\xi: F(x, \xi) \ge 0] \ge 1 - \epsilon \}$ \\
  $\bullet$ Propose a deterministic conservative approximation of $Z_D$. \\
  \\ \hline \hline
\end{longtblr}

\chapter{Chance-constrained optimization under limited distributional information: A review of reformulations based on sampling and distributional robustness}
This is a summary of \cite{kucukyavuza2022chance}.


\textbf{The chance-constrained problem CCP} In a probaiblity space $(\Omega, \mathcal{F}, \mathbb{P}^0)$
\begin{subequations} \label{CCP}
\begin{align}
  \min_x & \quad  c^T x \\
  & \text{s.t.} \quad \mathbb{P}^0 (x \in \mathcal{P} (\omega)) \ge 1 - \epsilon \\
  &  x \in \mathcal{X},
\end{align}
\end{subequations}
where $\omega$ is a random variable vector with a distribution $\\mathbb{P}^0$, $\mathcal{X} \subset \R^n$ is compact set defined by the determinitic constraints on the decision variables $x$ and risk level $\epsilon \le 0.05$.

\textbf{Properties of CCPs} In this survey, the study is discussing on the linear chance constrained problems, i.e. polyhedral $\mathcal{P}(\omega)$. 
More precisely, let 
\begin{equation}
  \mathcal{P}(\omega) := \{ x: T(\omega) x \ge r (\omega) \},
\end{equation}
where $T(\omega)$ is an $m \times n$ matrix  of random constraint coefficients and $r(\ranvar) \in \R^m$ is a vector of random right-hand sides.
\begin{itemize}
  \item $m=1:$ CCP is called \textit{individual} CCP.
  \item $m >1:$ CCP is called \textit{joint} CCP.
\end{itemize}
If for all $\omega \in \Omega, T(\omega) = T$, then the CCP has right-hand side uncertainty.
In contrast, the $technology$ $matrix$ $T(\omega)$ is random, the CCP has left-hand side uncertainty.
One of the reformulations is choosing the smaple space $\Omega$ to be finite, which means that the CCP is under finite discrete distribution.\\

\textbf{CCPs under finite discrete distribution} Given the probability space $(\Omega, 2^\Omega, \mathbb{P})$
The Sample Average Approach (SAA) of \eqref{CCP} is 
\begin{subequations} \label{CCP:SAA}
  \begin{align}
    \min_x & \quad  c^T x \\
    & \text{s.t.} \quad \frac{1}{N} \sum_{i \in [N]} \mathbbm{1} (x \notin \mathcal(\omega_i)) \le \epsilon, \\
    &  x \in \mathcal{X},
  \end{align}
  \end{subequations}
where $\mathbbm{1}(\cdot)$ is the indicator function.

$\clubsuit$ This leads to MIP via the introduction of binary variables and big-M constraints.

\begin{enumerate}
  \item[1.] \textbf{RHS uncertainty} 
The problem \eqref{CCP:SAA} becomes 
\begin{subequations}
\begin{align}
  \min_{x,t,z} & \quad c^T x \\
  & \quad \text{s.t.} x \in \mathcal{X}, Tx = \bar{r} + t \\
  & t_j \ge r_{i,j}(1-z_i), \forall i \in [N], \forall j \in [m] \\
  & \frac{1}{N} \sum_{i \in [N]} z_i \le \epsilon \\
  & t \in \mathbb{R}_+^m \\
  & z \in \{0,1\}^N,
\end{align}
\end{subequations}
where $\bar{r} \in \mathbb{R}^m$ is chosen to satisfy $r(\omega_i) \ge \bar{r}, \forall i $ and $r_i = (r_{i,1}, \dots, r_{i,m})^T$ denotes $r(\omega_i) - \bar{r}$. $[\dots]$

  \item[2.] \textbf{LHS uncertainty} 
The problem \eqref{CCP:SAA} becomes
\begin{subequations}
\begin{align}
  \min_{x,z} & \quad c^T x \\
  & \text{s.t.} x \in \mathcal{X} \\
  &  T(\omega_i)x \ge r(\omega_i) - M(\omega_i) z_i, \forall \in [N] \label{ConLHS}\\
  & \frac{1}{N} \sum_{i \in [N]} z_i \le \epsilon \\
  & z \in \{0,1 \}^N,
\end{align}
\end{subequations}
where $M(\omega_i), i \in [N]$ is a vector of big-M coefficients such that when $z_i=1$, inequality \eqref{ConLHS} is redundant.
\end{enumerate}





\bibliographystyle{plainnat}
\bibliography{con-literature.bib}



\end{document}